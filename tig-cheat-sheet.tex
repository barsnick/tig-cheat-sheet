\documentclass[a4paper,10pt,landscape]{article}
\usepackage{multicol}
\usepackage{ifthen}
\usepackage[landscape]{geometry}
\usepackage{hyperref}

% Build a pdf online with:
% http://latex.informatik.uni-halle.de/latex-online/latex.php?spw=2&id=626837_hRLskRF34GHL

% To make this come out properly in landscape mode, do one of the following
% 1.
%  pdflatex latexsheet.tex
%
% 2.
%  latex latexsheet.tex
%  dvips -P pdf  -t landscape latexsheet.dvi
%  ps2pdf latexsheet.ps

% This sets page margins to .5 inch if using letter paper, and to 1cm
% if using A4 paper. (This probably isn't strictly necessary.)
% If using another size paper, use default 1cm margins.
\ifthenelse{\lengthtest { \paperwidth = 11in}}
	{ \geometry{top=.5in,left=.5in,right=.5in,bottom=.5in} }
	{\ifthenelse{ \lengthtest{ \paperwidth = 297mm}}
		{\geometry{top=1cm,left=1cm,right=1cm,bottom=1cm} }
		{\geometry{top=1cm,left=1cm,right=1cm,bottom=1cm} }
	}

% Turn off header and footer
\pagestyle{empty}
 

% Redefine section commands to use less space
\makeatletter
\renewcommand{\section}{\@startsection{section}{1}{0mm}%
                                {-1ex plus -.5ex minus -.2ex}%
                                {0.5ex plus .2ex}%x
                                {\normalfont\large\bfseries}}
\renewcommand{\subsection}{\@startsection{subsection}{2}{0mm}%
                                {-1explus -.5ex minus -.2ex}%
                                {0.5ex plus .2ex}%
                                {\normalfont\normalsize\bfseries}}
\renewcommand{\subsubsection}{\@startsection{subsubsection}{3}{0mm}%
                                {-1ex plus -.5ex minus -.2ex}%
                                {1ex plus .2ex}%
                                {\normalfont\small\bfseries}}
\makeatother

% Don't print section numbers
\setcounter{secnumdepth}{0}


\setlength{\parindent}{0pt}
\setlength{\parskip}{0pt plus 0.5ex}


% -----------------------------------------------------------------------

\begin{document}

\raggedright
\footnotesize
\begin{multicols}{3}


% multicol parameters
% These lengths are set only within the two main columns
%\setlength{\columnseprule}{0.25pt}
\setlength{\premulticols}{1pt}
\setlength{\postmulticols}{1pt}
\setlength{\multicolsep}{1pt}
\setlength{\columnsep}{2pt}

\begin{center}
     \Large{\textbf{Tig Cheat Sheet}} \\
\end{center}

\section{Switch between views}
\begin{tabular}{@{}ll ll ll@{}}
\verb!m! & \verb!Main!. & \verb!p! & \verb!Pager!. & \verb!t! & \verb!Directory! tree.\\
\verb!d! & \verb!Diff!. & \verb!r! & \verb!Refs!. & \verb!f! & \verb!File! blob.\\
\verb!l! & \verb!Log!. & \verb!S! & \verb!Status!. & \verb!y! & \verb!Stash!.\\
\verb!B! & \verb!Blame!. & \verb!c! & \verb!Stage!. & \verb!h! & \verb!Help!.\\
\end{tabular}

\section{Cursor Navigation}
\begin{tabular}{@{}ll@{}}
\verb!k! / \verb!j! & Move cursor one line \verb!up! / \verb!down!. \\
\verb!PgUp,-,a! & Move cursor one page up. \\
\verb!PgDown, Space! & Move cursor one page down. \\
\verb!Home! / \verb!End! & Jump to \verb!first! / \verb!last! line. \\
\end{tabular}

\section{View Manipulation}
\begin{tabular}{@{}lp{6.5cm}@{}}
\verb!q! & Close current view, go to previous one until last. \\
\verb!Q! & Quit tig. \\
\verb!Enter! & Split the view to show the commit diff / In the diff view, scroll the view one line down. \\
\verb!Tab! & Switch to next view. \\
\verb!R! & Reload and refresh the current view. \\
\verb!O! & Maximize the current view to fill the whole display. \\
\verb!Up! & Move the cursor one line up. In diff view from the main view, point to the previous commit and update the diff view to display it. \\
\verb!Down! & Similar to Up but will move down. \\
\verb!,! & Move to parent. \\
\end{tabular}

\section{Scrolling}
\begin{tabular}{@{}ll@{}}
\verb!Insert! / \verb!Delete! & Scroll view one line \verb!up! / \verb!down!. \\
\verb!w! / \verb!s! & Scroll view one page \verb!up! / \verb!down!. \\
\verb!Left! / \verb!Right!& Scroll view one column \verb!left! / \verb!right!. \\
\verb!|! & Scroll view to the first column. \\
\end{tabular}

\section{Searching}
\begin{tabular}{@{}ll@{}}
\verb!/! & Search the view using RegExp. \\
\verb!?! & Search backwards. \\
\verb!n! / \verb!N! & Find \verb!next! / \verb!previous! match. \\
\end{tabular}

\section{View Specific Actions}
\begin{tabular}{@{}lp{6.5cm}@{}}
\verb!u! & Update status of file.\\
\verb!M! & Resolve unmerged file by launching git-mergetool. \\
\verb#!# & Checkout file (reset its content). \\
\verb!1! & Stage single diff line. \\
\verb!@! & Move to next chunk in the stage view. \\
\verb!]! / \verb![! & \verb!Increase! / \verb!Decrease! the diff context. \\
\end{tabular}

\section{Main View (m)}
\begin{tabular}{@{}lp{6.5cm}@{}}
\verb!C! & Cherry-pick the current commit (e.g. git cherry-pick  \%(commit)). \\
\end{tabular}

\section{Diff View (d)}
\begin{tabular}{@{}lp{6.5cm}@{}}
\verb!k! / \verb!j! & Move cursor one line \verb!up! / \verb!down! \\
\verb!Up! / \verb!Down! & Go to \verb!previous! / \verb!next! commit and update the diff view. \\
\end{tabular}

\section{Blame View (B)}
\begin{tabular}{@{}lp{6.5cm}@{}}
\verb!,! & Load blame for the parent commit (Parent is queried for merges). \\
\end{tabular}

\section{Refs View(r)}
\begin{tabular}{@{}lp{6.5cm}@{}}
\verb!Enter! & Open commit log view. \\
\verb!C! & Checkout current branch (e.g. git checkout \%(branch)). \\
\end{tabular}

\section{Status View (S)}
\begin{tabular}{@{}lp{6.5cm}@{}}
\verb!u! & Add or stage file.\\
\verb!M! & Resolve unmerged file by launching git-mergetool. \\
\verb#!# & Checkout file (reset its content). \\
\verb!1! & Stage single diff line. \\
\verb!@! & Move to next chunk in the stage view. \\
\verb!]! / \verb![! & \verb!Increase! / \verb!Decrease! the diff context. \\
\verb!C! & Commit (e.g. git commit) \\
\end{tabular}

\section{Stage View (c)}
\begin{tabular}{@{}lp{6.5cm}@{}}
\verb!u! & Stage only chunk for next commit (on a diff chunk), otherwise stage all changes. \\
\end{tabular}

\section{Tree View (t)}
\begin{tabular}{@{}lp{6.5cm}@{}}
\verb!,! & Switch to the parent directory. \\
\end{tabular}

\section{Prompt (:)}
\begin{tabular}{@{}ll@{}}
\textbf{:\textless number\textgreater} & Jump to the specific line number, e.g. :80. \\
\textbf{:\textless sha\textgreater} & Jump to a specific commit, e.g. :2f12bcc. \\
\textbf{:\textless x\textgreater} & Execute the corresponding key binding, e.g. :q. \\
\textbf{:!\textless cmd\textgreater} & Run a command, e.g. :!git log -p. \\
\textbf{:\textless action\textgreater} & Execute a Tig command, e.g. :edit. \\
\end{tabular}

\section{Misc / Options}
\begin{tabular}{@{}lp{6.5cm}@{}}
\verb!r! & Redraw screen. \\
\verb!z! & Stop all background loading (a repository with a long history without limiting the revision log) \\
\verb!v! & Show version. \\
\verb!o! & Open option menu \\
\verb!.! & Toggle line numbers on/off. \\
\verb!D! & Toggle date display on/off/short/relative/local. \\
\verb!A! & Toggle author display on/off/abbreviated /email/email user name. \\
\verb!g! & Toggle revision graph visualization on/off. \\
\verb!~! & Toggle (line) graphics mode \\
\verb!F! & Toggle reference (tag and branch) display on/off. \\
\verb!W! & Toggle ignoring whitespace on/off for diffs \\
\verb!X! & Toggle commit ID display on/off \\
\verb!%! & Toggle file filtering in order to see the full diff instead of only the diff concerning the currently selected file. \\
\verb!$! & Toggle highlighting of commit title overflow. \\
\verb!:! & Open prompt to run a command. \\
\verb!e! & Open file in editor. \\
\verb!G! & Run garbage collection (e.g. git gc) \\
\end{tabular}

\section{External Commands}
Provide a way to easily execute a script or program.
Bound to keys and use information from the current browsing state (e.g current commit ID).

Built-in external commands:
\begin{tabular}{@{}lll@{}}
\verb!main! & \verb!C! & git cherry-pick  \%(commit)\\
\verb!status! & \verb!C! & git commit \\
\verb!generic! & \verb!G! & git gc \\
\end{tabular}


\section{Set your custom bind Commands}

In the .gitconfig file, under the \verb![tig "bind"]! section, use pattern:
keymap = key action

\begin{tabular}{@{}lp{6.5cm}@{}}
\verb!Keymaps! & main, diff, log, help, pager, status, stage, tree, blob, blame, branch, and generic (global keymap)\\
\verb!Special Keys! & Enter, Space, Backspace, Tab, Escape, Left, Right, Up, Down, Insert, Delete, Hash, Home, End, PageUp, PageDown, F1,..., F12\\
\end{tabular}
note: use \^{} to set a keymap with the ctrl key (\^{}j for ctrl-j)

\subsection{External command control flags:}
\begin{tabular}{@{}lp{6.5cm}@{}}
\verb!@! & Run the command in background (no output).\\
\verb!?! & Prompt the user before running the command.\\
\verb!<! & Exit Tig after running the command.\\
\end{tabular}


\subsection{Browsing state variables:}
\begin{tabular}{@{}lp{6.5cm}@{}}
\verb!%(head)! & Currently viewed head ID. Defaults to HEAD\\
\verb!%(commit)! & Currently selected commit ID.\\
\verb!%(blob)! & Currently selected blob ID.\\
\verb!%(branch)! & Currently selected branch name.\\
\verb!%(stash)! & Currently selected stash name.\\
\verb!%(directory)! & Current directory path in the tree view; empty for the root directory.\\
\verb!%(file)! & Currently selected file.\\
\verb!%(ref)! & Reference given to blame or HEAD if undefined.\\
\verb!%(revargs)! & Revision arguments passed on the command line.\\
\verb!%(fileargs)! & File arguments passed on the command line.\\
\verb!%(diffargs)! & Diff options passed on the command line.\\
\verb!%(prompt)! & Prompt for the argument value.\\
\end{tabular}

\begin{verbatim}
ex:
[tig "bind"]
    # 'unbind' the default quit key binding
    main = Q none
    # Cherry-pick current commit onto current branch
    generic = C !git cherry-pick %(commit)
    # Amend last commit in status view
    #(prompting user and exit after)
    status = A !?<git commit --amend
\end{verbatim}

For more informations on custom external commands, see \url{http://jonas.nitro.dk/tig/tigrc.5.html}


\rule{0.3\linewidth}{0.25pt}
\scriptsize

tig project:\url{https://github.com/jonas/tig}
\linebreak
tig manual:\url{http://jonas.nitro.dk/tig/manual.html}
\linebreak
CC BY-SA P. Miossec \url{https://github.com/pmiossec/tig-cheat-sheet}

\end{multicols}
\end{document}
